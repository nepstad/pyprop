\documentclass[a4paper]{article}

\usepackage{graphicx}


%opening
\title{Notes on $HD^+$ Propagation}
\author{Tore Birkeland}

\begin{document}

\maketitle

\section{The Problem}
We wish to simulate certain diatomic ionic molecules ($HD^+$, $H_2^+$, $D_2^+$) under a femtosecond laser pulse. 

\subsection{Center of mass Coordinates}
A diatomic molecule has the following hamiltonian in the laboaratory frame in atomic units
\begin{equation}
 H = - \frac{1}{2 M_1} \nabla^2_{R_1} - \frac{1}{2 M_2} \nabla^2_{R_2} - \frac{1}{2 m_e} \nabla^2_{r_e} + \frac{Z_1 Z_2}{|R_1 - R_2|} - \frac{Z_1}{|r_e - R_1|} - \frac{Z_2}{|r_e - R_2|}
\end{equation}
Which can be transformed into center of mass coordinates by the transformation
\begin{eqnarray}
	R_c &=& \frac{M_1 R_1  + M_2 R_2 + m_e r_e}{M_c} \\
	R &=& R_1 - R_2 \\
	r &=& r_e - \frac{M_1 R_1 + M_2 R_2}{M_1 + M_2}
\end{eqnarray}
The Hamiltonian in the new coordinates is then
\begin{equation}
\label{eqn:full-hamiltonian-cm}
 H = - \frac{1}{2 M_c} \nabla^2_{R_c} - \frac{1}{2 M_n} \nabla^2_{R_n} - \frac{1}{2 \mu} \nabla^2_{r} + \frac{Z_1 Z_2}{|R|} - \frac{Z_1}{|r - \rho_1 R|} - \frac{Z_2}{|r - \rho_2 R|}
\end{equation}
where
\begin{eqnarray*}
 M_c &=& M_1 + M_2 + m_e \\
 M_n &=& \frac{M_1 M_2}{M_1 + M_2} \\
 \mu &=& \frac{m_e(M_1 + M_2)}{M_c} \\
 \rho_i &=& \frac{M_i}{M_1 + M_2} 
\end{eqnarray*}

\subsection{Born Openheimer Approximation}
We will neglect the effect of the laser on the center-of-mass system, and therefore the first term in \ref{eqn:full-hamiltonian-cm}, will separate out, and the system is reduced from a nine to six dimensions. Furthermore, we will assume the Born-Openheimer approximation, in which one assumes that movement in the $r$ direction will be instantaneous compared to movement in the $R$ direction, and one can therefore separate the equation into two three dimensional equations.
\begin{equation}
	\Psi(R, r) = \psi(R) \phi(r; R)
\end{equation}
\begin{equation}
	H(R, r) = H_n(R) + H_e(r; R)
\end{equation}
Where
\begin{eqnarray}
	H_n(R) &=& - \frac{1}{2 M_n} \nabla^2_{R_n} + \frac{Z_1 Z_2}{|R|} \\
	H_e(r; R) &=&  - \frac{1}{2 \mu} \nabla^2_{r} - \frac{Z_1}{|r - \rho_1 R|} - \frac{Z_2}{|r - \rho_2 R|}
\end{eqnarray}

One first solves the electronic system paramterically for R
\begin{equation}
	\label{eqn:electron-hamiltonian-bo}
	H_e(r; R) \phi_i(r; R) = \epsilon_i(R) \phi_i(r; R)
\end{equation}
This gives a set of energy curves $\epsilon(R)$, which can be used to solve the the nuclear system
for each electronic state $i$
\begin{equation}
	\label{eqn:nuclear-hamiltonian-bo}
 	H_n^i(R) \psi_{i,j}(R) = \left( H_n(R) + \epsilon_i(R) \right) \psi_{i,j}(R) = E_{i,j} \psi_{i,j}(R)
\end{equation}

\subsection{Laser Interaction}
We will model the laser field with the usual semi classical model. For a polarized laser pulse in 
the length gauge, this is in the laboratory frame
\begin{equation}
 H_l = e E(t) \varepsilon ( r - Z_1 R_1 - Z_2 R_2)
\end{equation}
Which in the center of mass frame is (with center of mass motion disregarded)
\begin{equation}
 H_l = E(t) \varepsilon ( \kappa_r r - \kappa_R R )
\end{equation}
\begin{eqnarray}
	\kappa_r &=& \frac{(Z_1 + Z_2) m_e + (M_1 + M_2)}{M_c} \\
	\kappa_R &=& \frac{Z_1 M_1 - Z_2 M_2}{M_1 + M_2]}
\end{eqnarray}
If the electronic system in the Born-Openheimer approximation is solved $\{ \phi_i(r; R) \}$, one can calculate the induced dipole moment 
between two electronic states $\phi_i(r; R)$ and $\phi_j(t; R)$
\begin{equation}
	d_{i,j}(R) = \int \phi_i^*(r; R) r \phi_j(r, R) dr
\end{equation}
If we put this and eqn \ref{eqn:nuclear-hamiltonian-bo} into the time dependent Schrödinger equation, we get
\begin{equation}
	\label{eqn:nuclear-hamiltonian-laser}
 - \imath \frac{\partial}{\partial t} \psi_i = \left( H_n + \epsilon_i(R) \right) \psi_i + E(t) \varepsilon \left( \kappa_r \sum_j d_{i, j}(R) \psi_j + \kappa_R R \psi_i\right)
\end{equation}

\subsection{1D Model}
Here, we will concentrate on propagating the nuclear equation. In accordance with eqn \ref{eqn:nuclear-hamiltonian-laser}, we do not need to know the exact electronic wavefunctions, as long as the induced dipole moments $d_{i,j}(R)$ are available. Furthermore, we will ignore rotational effects in the molecule, effectively reducing the problem from a full 3D problem to a much simpler 1D model.

As the difference The dipole moments for the two lowest electronic states was calculated by Bates (REF), and the following expression was found to be a good approximation for the dipole moments
\begin{equation}
	d(R) = -\frac{1}{2 + 1.4 |R|} + \frac{|R|}{2 \sqrt{1 - \rho(R)^2}}
\end{equation}
where 
\begin{equation}
	\rho(R) = (1 + |R| + |R|^{2/3}) exp(-|R|) 
\end{equation}
As we will only perform calculations on the radial part of the nuclear wavefunction, we can simplify $H_1$
\begin{equation}
	H_1 = \frac{1}{2 M_n} \left( \frac{\partial^2}{\partial R^2} + \frac{2}{R} \frac{\partial}{\partial R} \right)
\end{equation}
To get rid of the first derivative term, will perform calculations on a reduced wavefunction
\begin{equation}
	\psi_{reduced}(R) = r^{-1} \psi_{original}(R)
\end{equation}
This transforms $H_1$ into
\begin{equation}
	H_1 = \frac{1}{2 M_n} \frac{\partial^2}{\partial R^2} 
\end{equation}
And $\psi_{reduced}(R)$ will be refered to as $\psi(R)$ below unless otherwise stated

\section{Numerical Method}
We will use a standard splitting technique for time marching the TDSE. The Hamiltonian is split in three parts
\begin{eqnarray}
	H_1 &=& \frac{1}{2 M_n} \nabla^2_R \\
	H_2 &=& \frac{Z_1 Z_2}{|R|} + \epsilon_i(R) \\
	H_3 &=& E(t) \varepsilon \left( \kappa_r \sum_j d_{i, j}(R) + \kappa_R R \right)
\end{eqnarray}
\begin{equation}
	- \imath \frac{\partial}{\partial t} \mathbf{\psi}(t) = (H_1 + H_2 + H_3) \mathbf{\psi}(t)
\end{equation}
A truncated Magnus expansion of the solution can be written as
\begin{equation}
 	\psi(t + \Delta t) = \exp( - \imath \Delta t (H_1 + H_2 + H_3) ) \psi(t) + O(\Delta t^2)
\end{equation}
Since each of the sub-Hamiltonians can be solved accurately (see below), it makes sense to write the solution in terms of the propagators for the individual sub-Hamiltonians
\begin{equation}
	\psi(t + \Delta t) = \exp(- \imath \Delta t H_1) \exp(- \imath \Delta t H_2) \exp(- \imath \Delta t H_3) \psi(t) + O(\Delta t^2)
\end{equation}
The accuracy of this splitting can be increased by considering other splitting schemes such as the symmetric Strang splitting scheme (REF)

As all parts of the Hamiltonian are hermitian, they are all orthogononaly diagonalizable. This means that an eigenvector decomposition can be very accurately computed.  The key to accurately propagating each sub-propagator, is then to find a space in which they are diagonal, or in other words if we choose a starting space, find a linear transformation that diagonalizes the operator.

\subsection{Propagating $H_1$}
We will use a discretized grid representation as our starting point. It is well known that the Fourier transformation $\bar{f}(k) = {\cal F}_R f(R)$ diagonalizes the differentiation operator $\partial^2/\partial R^2$
\begin{equation}
	\frac{\partial^2 f(R)}{\partial R^2} = {\cal F}^-1 -k^2 {\cal F} f(R)
\end{equation}
For a periodic equispaced discretized grid, the discrete Fourier transfom $F$ can be used to approximate to the differentiation. Normally, when dealing with radial grids, one must take care to get the correct $\lim_{R \rightarrow 0} \psi(R) = 0$ behaviour. In our case, however, the steep positive potential as $R \rightarrow 0$, automatically keeps the wavefunction small near the origin, taking away the need for anti symmetrization or other techniques otherwise employed.

As there are no coupling between the different electronic states, we can transform both electronic states to Fourier space, perform the differentiation, and transform back.
\begin{equation}
 	\psi_i^{H_1}(R, t+\Delta t) = F^{-1} \exp \left( \imath \Delta t \frac{k^2}{2 M_n} \right) F \psi_i(R, t)
\end{equation}

\subsection{Solving $H_2$}
\begin{equation}
	H_2 = \frac{Z_1 Z_2}{|R|} + \epsilon_i(R) 
\end{equation}
$H_2$ is a grid function, and therefore diagonal in the grid space. We can therefore directly solve write up the sub propagator, using a different potential 
for each electronic wavefunction
\begin{equation}
 	\psi_i^{H_2}(R, t+\Delta t) = \exp \left\{ \imath \Delta t \left(\frac{Z_1 Z_2}{|R|} + \epsilon_i(R) \right)  \right\} \psi_i(R, t)
\end{equation}

\subsection{Solving $H_3$}
Using a two level approximation for the electronic states, the electronic coupling part of the Hamiltonian looks like this
\begin{equation}
 	H_3 = E(t) D(R)
\end{equation}
Where
\[
	D(R) = \left(\begin{array}{cc}
			\kappa_R R & \kappa_r d_{1,2}(R) \\
			\noalign{\medskip}
			\kappa_r d_{1,2}^*(R) & \kappa_R R
		\end{array}
		\right)
\]
$H_3$ is a $2x2$ matrix coupling the two electronic states for each value of $R$. However, there are no coupling between different values of $R$. We can 
therefore find a transformation $G(R)$ diagonalizes the electronic coupling. As $D(R)$ is Hermitian, this transformation will be orthogonal.
\begin{equation}
	D(R) = G(R) \Lambda(R) G^*(R)
\end{equation}
This gives us a way to propagate $H_3$
\begin{equation}
 	\psi_i^{H_3}(R, t+\Delta t) = G(R) \exp \left\{ \imath \Delta t E(t) \Lambda(R) \right\} G^*(R) \psi_i(R, t)
\end{equation}

\section{Results}
\begin{figure}[p]
 \centering
 \includegraphics[width=12cm]{fig/delay_scan_h2p.eps}
 \caption{Eigenstate population for $H_2^+$ as a function of pulse delay time for a $14fs$ pulse of intensity $2e14 W/cm^2$.}
 \label{fig:delay_scan_h2+}
\end{figure}

\begin{figure}[p]
 \centering
 \includegraphics[width=12cm]{fig/delay_scan_d2p.eps}
 \caption{Eigenstate population for $D_2^+$ as a function of pulse delay time for a $14fs$ pulse of intensity $2e14 W/cm^2$.}
 \label{fig:delay_scan_d2+}
\end{figure}

\begin{figure}[p]
 \centering
 \includegraphics[width=12cm]{fig/delay_scan_hdp.eps}
 \caption{Eigenstate population for $HD^+$ as a function of pulse delay time for a $14fs$ pulse of intensity $2e14 W/cm^2$.}
 \label{fig:delay_scan_hd+}
\end{figure}

% \begin{figure}[p]
%  \centering
%  \includegraphics[width=12cm]{fig/delay_scan_hdp_nodipole.eps}
%  \caption{Eigenstate population for $HD^+$ as a function of pulse delay time for a $14fs$ pulse of intensity $2e14 W/cm^2$ 
% without the static dipole moment taken into consideration}
%  \label{fig:delay_scan_d2+_nodipole}
% \end{figure}

\begin{figure}[p]
 \centering
 \includegraphics[width=12cm]{fig/delay_scan_nodipole_diff.eps}
 \caption{Difference between calculations with and without the static dipole term included for $HD^+$}
 \label{fig:delay_scan_hd+_nodipole_diff}
\end{figure}

Figures \ref{fig:delay_scan_h2+}-\ref{fig:delay_scan_hd+} shows the population of different eigenstates as a function of different pulse delay times $t_d$. We notice that all the graphs show simiar features, in that the ionization probability is oscillating with the first minima at $t_d \approx 20fs$. Furthermore, we notice that the oscillations have different frequencies for the three molecules, which can be accredited to the different masses of the systems (Which is the only difference between $H_2^+$ and $D_2^+$). The $HD^+$ molecule has also a static dipole, which potentially could have changed the dynamics, but as seen in Figure \ref{fig:delay_scan_hd+_nodipole_diff}, it has no real effect.

Another noticable effect is that for 

\end{document}