
\chapter{Installation}
\section{Prerequistes}
Pyprop relies heavily on C++ templates, so a fairly standards compliant C++ compiler is required.
It is developed using mainly GNU C++ and Intel C++, but other compliant compilers should work as well.
Python 2.4 or newer is highly recommended. Pyprop might work with v2.2 or 2.3, but it is only tested on
Python 2.4 and 2.5. Pyprop also uses pyste (a part of boost::python) to generate boost::python code. 
A modified version of pyste must be installed in order to successfully compile pyprop.

\subsection{Using Tores prerequisites}
On systems yours truly are using, the required programs and libraries are likely to be 
installed, and you may use them instead of installing your own copies. In most cases, everything
will be installed in a folder called \textit{sys} located in torebi's home folder. Assuming bash
is the shell used, add the following to the end of your \textit{~/.bashrc}
\begin{bash}
export TOREHOME=~torebi
export PATH=$TOREHOME/sys/bin:$PATH
export LD_LIBRARY_PATH=$TOREHOME/sys/lib:$LD_LIBRARY_PATH
export PYTHONPATH=$TOREHOME/sys/lib/python
export CFLAGS="-I$TOREHOME/sys/include -fPIC"
export CXXFLAGS="-fPIC"
export CPPFLAGS="-I$TOREHOME/sys/include"
export LDFLAGS="-L$TOREHOME/sys/lib"
export BOOST_ROOT="$TOREHOME/download/lib/boost_1_33_1"
$
\end{bash}

\subsubsection*{Linux (fimm.bccs.uib.no)}
On fimm, you must also choose a compiler. The Intel C++ compiler produces blitz++ code that is serveral
times faster than GNU C++, so it is recommended to use that compiler. To use Intel C++ along with
OpenMPI, add the following lines to your \textit{~/.bashrc}
\begin{bash}
source ~/use_compiler intel > /dev/null
source ~/use_mpi openmpi > /dev/null	
\end{bash}
	
\subsubsection*{Other platforms (AIX)}
I have yet to find a sane AIX platform. On njord.hpc.ntnu.no, the installed python version appears to be 32bit, 
and as such, completely useless for any problems of reasonable size (and it creates a lot of problems for other modules
trying to be 64 bit). So until someone is able to compile a properly working 64bit python on AIX, pyprop will not work
on that platform. Feel free to apply pressure on hpc.uib.no by filing a request to support-notur@
	
\subsection{Installing your own prerequisites}
Installation of the below mentioned applications, libraries and modules can be slightly tricky, and I 
will not cover the details here, however pyste and blitz++ must be patched before they can be used 
with pyprop. The patch files can be found in the \textit{./patch/} subdirectory in the pyprop 
source directory. There are some more details concerning the installation in the INSTALL textfile
in the pyprop folder.

\subsubsection*{Required applications}
\begin{itemize}
\item Standards compliant C++ compiler (GNU C++, Intel C++, etc.)
\item Python 2.4 or newer (\url{http://www.python.org})
\item GCC\_XML, cvs version (\url{http://www.gccxml.org})
\item CMake 2.4 or newer (\url{http://www.cmake.org})
\end{itemize}
	
\subsubsection*{Required libraries}
The following libaries must be compiled and available in order to compile pyprop
\begin{itemize}
\item blitz++ v0.9 (\url{http://www.oonumerics.org/blitz})
\item boost::python v1.33.1 (\url{http://www.boost.org})
\item FFTW 3.x (\url{http://www.fftw.org})
\item BLAS (ATLAS or some other implementation supporting the cblas interface) (\url{http://math-atlas.sourceforge.net/})
\item LAPACK (\url{http://netlib.org/lapack})
\end{itemize}
ATLAS and LAPACK can easily be compiled into one library, which simplifies things considerably (especially the compilation
of numpy). See the ATLAS documentation for more info.

\subsubsection*{Required Python modules}
The following python modules must be installed and available in order to run pyprop:
\begin{itemize}
\item numpy v1.0 or later (\url{http://numpy.scipy.org/})
\item matplotlib 0.87.7 or later (\url{http://matplotlib.sourceforge.net/})
\item elementtree (\url{http://effbot.org/downloads/\#elementtree})
\end{itemize}
	
\subsubsection*{Recommended Python modules}
In addition the following modules are highly recommended:
\begin{itemize}
\item ipython (\url{http://ipython.scipy.org}) (A great interactive python interpreter)
\item scipy (\url{http://www.scipy.org}) (Library for almost everything related to scientific computing)
\item pymat (\url{http://claymore.engineer.gvsu.edu/~steriana/Python/pymat.html}) (MATLAB interface to Python. 
	Makes it possible to do plotting and analyisis in MATLAB directly from Python)
\end{itemize}

\section{Compilation}
\subsection*{Getting the source tree}
Locate the folder where you whish to install pyprop. This folder will be referenced to as the pyprop folder.
First you must get the newest version from the subversion repository. The project is hosted on Google Code, and 
can be accessed from subversion (svn). In your console window, run the command
\begin{verbatim}
	svn co http://pyprop.googlecode.com/svn/trunk pyprop
\end{verbatim}
This will download the entire pyprop source code into a folder called pyprop. Later, you can download the latest
changes by typing
\begin{verbatim}
	svn update
\end{verbatim}
from the pyprop folder.


\subsection*{Configuring Makefiles}
When the pyprop is downloaded, the platform specific makefile must be configured. First make
a copy of the example makefile provided
\begin{verbatim}
	cp Makefile.platform.example Makefile.platform
\end{verbatim}
and open Makefile.platform in your favourite editor. This is the only makefile you will need to edit. 
Changes will be reflected in the pyprop core module as well as in all plugins compiled with the provided
makefiles (on fimm, you may use \textit{core/makefiles/Makefile.platform.fimm} as your starting point).

The parameter MPICXX must be set to the compiler you whish to use. Unless -DINGLEPROC is set, this compiler
must be able to compile MPI programs. Until the MPI-support has been properly fixed, it is strongly recommended
to use \-DSINGLEPROC and a standard compiler like \textit{g++} or \textit{icpc}.
The flag -DLONGSIZE\_32BIT be set if and only if pyprop is compiled under a 32bit operating system.

For debugging, the following CPPFLAGS are recommended
\begin{verbatim}
	CPPFLAGS := $(CPPFLAGS) -DSINGLEPROC -DBZ_DEBUG -DPYPROP_DEBUG
\end{verbatim}
For optimization, -DBZ\_DEBUG and -DPYPROP\_DEBUG should not be used, and the following
CXXFLAGS is suitable for the GNU C++ Compiler
\begin{verbatim}
	CXXFLAGS := $(CXXFLAGS) -Wall -O3 -ffast-math -funroll-loops
\end{verbatim}
For the Intel C++ Compiler, the following optimization flags are commonly used
\begin{verbatim}
	CXXFLAGS := -$(CXXFLAGS) -O3 -Zp16 -ip -ansi_alias -no-ipo
\end{verbatim}

\subsection*{Compiling}
Compilation can be VERY time consuming, taking up to 20 minutes and consuming 200+MB RAM. 
The most time consuming part of the compilation is when pyste generates boost::python wrappers, and
it is generally recommended to drink some coffee while waiting.
\begin{verbatim}
	make clean
	make pyclean
	make
\end{verbatim}
The first time pyprop is built, it may complain about not knowing how to create certain .cpp files from
.d files. In that case, just run \textit{make} again, and the problem should go away.

Normally the autodependencies will take care of when recompilation of certain modules are needed, 
such that the user will only need to type \textit{make}, and never \textit{make clean}. However, if
some of the interfaces are changed, that is some public propery or method is added or changed, a pyste
recompilation might be needed. In that case the user should do a complete remake to ensure that everything
is up to date.
\begin{verbatim}
	make clean
	make pyclean
	make
\end{verbatim}

\section{Testing}
Go to the pyprop folder, and run the command
\begin{verbatim}
	python -c "import pyprop"
\end{verbatim}
If that does not work it means that pyprop has not been properly compiled. If it works, you may go 
on to test some examples

