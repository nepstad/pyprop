
\chapter{Installation}
\section{Prerequistes}
Pyprop relies heavily on C++ templates, so a fairly standards compliant C++ compiler is required.
It is developed using mainly GNU C++ and Intel C++, but other compliant compilers should work as well.
Python 2.4 or newer is highly recommended. Pyprop might work with v2.2 or 2.3, but it is only tested on
Python 2.4 and 2.5. Pyprop also uses pyste (a part of boost::python) to generate boost::python code. 
A modified version of pyste must be installed in order to successfully compile pyprop.

\textbf{An alternative to this procedure is to use someone elses prerequisites... add more info on that later}

The following libaries must be compiled and available in order to compile pyprop
\begin{itemize}
\item blitz++ v0.9 (http://www.oonumerics.org/blitz)
\item boost::python v1.33.1 (http://www.boost.org)
\item BLAS (ATLAS or some other implementation supporting the cblas interface) (http://math-atlas.sourceforge.net/)
\item LAPACK (http://netlib.org/lapack)
\end{itemize}
ATLAS and LAPACK can easily be compiled into one library, which simplifies things considerably. See the ATLAS documentation
for more info.

The following python modules must be installed and available in order to run pyprop:
\begin{itemize}
\item numpy v1.0 or later (http://numpy.scipy.org/)
\item matplotlib 0.87.7 or later (http://matplotlib.sourceforge.net/)
\end{itemize}
	
In addition the following modules are highly recommended:
\begin{itemize}
\item ipython (http://ipython.scipy.org) (A great interactive python interpreter)
\item scipy (http://www.scipy.org) (Library for almost everything related to scientific computing)
\item pymat (http://claymore.engineer.gvsu.edu/~steriana/Python/pymat.html) (MATLAB interface to Python. 
	Makes it possible to do plotting and analyisis in MATLAB directly from Python)
\end{itemize}

Installation of the mentioned libraries and modules can be slightly tricky, and I will not cover the details here, 
however pyste and blitz++ must be patched before they can be used with pyprop. The patch files can be found in 
the \textit{./patch/} subdirectory in the pyprop source directory.

\section{Compilation}
\subsection*{Getting the source tree}
Locate the folder where you whish to install pyprop. This folder will be referenced to as the pyprop folder.
First you must get the newest version from the subversion repository. UoB has no working subversion repository at
the moment, so you must connect to my server (apokalypse.no). Username is provided on request to 
tore.birkeland@math.uib.no. 
\begin{verbatim}
	svn co http://apokalypse.no:81/svn/repo/pyprop/trunk .
\end{verbatim}
This will download the entire pyprop source code into the current directory. Later, you can get the latest
changes by typing
\begin{verbatim}
	svn update
\end{verbatim}
in the same folder.

\subsection*{Configuring Makefiles}
When the pyprop is downloaded, the platform specific makefile must be configured. First make
a copy of the example makefile provided
\begin{verbatim}
	cp Makefile.platform.example Makefile.platform
\end{verbatim}
and open Makefile.platform in your favourite editor. This is the only makefile you will need to edit. 
Changes will be reflected in the pyprop core module as well as in all plugins compiled with the provided
makefiles.

The parameter MPICXX must be set to the compiler you whish to use. Unless -DINGLEPROC is set, this compiler
must be able to compile MPI programs. Until the MPI-support has been properly fixed, it is strongly recommended
to use \-DSINGLEPROC and a standard compiler like \textit{g++} or \textit{icpc}.
The flag -DLONGSIZE\_32BIT be set if and only if pyprop is compiled under a 32bit operating system.

For debugging, the following CPPFLAGS are recommended
\begin{verbatim}
	CPPFLAGS := $(CPPFLAGS) -DSINGLEPROC -DBZ_DEBUG -DPYPROP_DEBUG
\end{verbatim}
For optimization, the following -DBZ\_DEBUG and -DPYPROP\_DEBUG should not be used, and the following
CXXFLAGS is suitable for the GNU C++ Compiler
\begin{verbatim}
	CXXFLAGS := $(CXXFLAGS) -Wall -O3 -ffast-math -funroll-loops
\end{verbatim}
For the Intel C++ Compiler, the following optimization flags is commonly used
\begin{verbatim}
	CXXFLAGS := -$(CXXFLAGS) -O3 -Zp16 -ip -ansi_alias -no-ipo
\end{verbatim}

\subsection*{Compiling}
Compilation can be VERY time consuming, taking up to 20 minutes and consuming 200+MB RAM. 
The most time consuming part of the compilation is when pyste generates boost::python wrappers, and
it is generally recommended to drink some coffee while waiting.
\begin{verbatim}
	make clean
	make pyclean
	make
\end{verbatim}
The first time pyprop is built, it may complain about not knowing how to create certain .cpp files from
.d files. In that case, just run \textit{make} again, and the problem should go away.


Normally the autodependencies will take care of when recompilation of certain modules are needed, 
such that the user will only need to type \textit{make}, and never \textit{make clean}. However, if
some of the interfaces are changed, that is some public propery or method is added or changed, a pyste
recompilation might be needed. In that case the user should do a complete remake to ensure that everything
is up to date.
\begin{verbatim}
	make clean
	make pyclean
	make
\end{verbatim}

\section{Testing}
Go to the pyprop folder, and run the command
\begin{verbatim}
	python -c "import pyprop"
\end{verbatim}
If that does not work it means that pyprop has not been properly compiled. If it works, you may go 
on to test some examples

