\documentclass[a4paper,10pt]{article}

\newcommand{\abs}[1]{\vert #1 \vert}
\newcommand{\bra}[1]{\langle #1 \vert}
\newcommand{\ket}[1]{\ensuremath{\vert #1 \rangle}}
\newcommand{\innerproduct}[2]{\ensuremath{\langle #1 \vert #2 \rangle}}
\newcommand{\braketnorm}[1]{\ensuremath{\langle #1 \vert #1 \rangle}}
\newcommand{\expectationvalue}[1]{\ensuremath{\langle #1 \rangle}}
\newcommand{\timediff}[1]{\ensuremath{#1'}}
\newcommand{\operator}[1]{\ensuremath{\hat{#1}}}
\newcommand{\totaldiff}[2][]{\ensuremath{\frac{d#1}{d{#2}}}}
\newcommand{\totaldifforder}[3][]{\ensuremath{\frac{d^{#3}#1}{d{#2}^{#3}}}}
\newcommand{\partialdifforder}[3][]{\ensuremath{\frac{\partial^{#3}#1}{\partial{#2}^{#3}}}}
\newcommand{\partialdiff}[2][]{\ensuremath{\frac{\partial#1}{\partial{#2}}}}

% Title Page
\title{Calculating Ground State Energy}
\author{Tore Birkeland}

\begin{document}
\maketitle

\begin{abstract}
How to calculate the ground state energy in pyprop.
\end{abstract}

\section{Finding the ground state}
(More or less stolen from Lene\&Raymond(2006))
Using a TDSE solver (like pyprop) one can find the ground state of the given system
using what is known in the litterature as imaginary time propagation. Imaginary time
propagation is in essence to change the problem from the TDSE (with a static potential)
to a diffusion equation, to which one wants to find the stationary (ground) state.

Effectively this is done by replacing $t$ by $\tau = - \imath t$ in the TDSE. Because
all operators in the TDSE are Hermitian, using imaginary time transforms the equation 
into a real diffusion equation
\begin{equation}
	\partialdiff{\tau} \ket{\psi(\tau)} = - H \ket{\psi(\tau)}
\end{equation}
With solution
\begin{equation}
	\ket{\psi(\tau)} = e^{- H \tau} \ket{\psi(0)}
\end{equation}
If we expand $\ket{\psi(0)}$ in the eigenstates of $H$, $\ket{phi_n}$, the solution
becomes
\begin{equation}
	\ket{\psi(\tau)} = \sum_n e^{- E_n \tau} \ket{\phi_n} 
\end{equation}
From this we see that as we propagate $\tau$, and keep $\ket{\psi(\tau)}$ normalized,
all states with energy higher than the ground state will be exponentially damped, thus
the solution will rapidly converge towards the ground state.

\section{Finding the ground state energy directly}
If the solution is converged to the ground state (applying one more time step does not 
change the solution significantly) at $\tau = \tau_c$, we may calculate the energy by propagating 
the solution one final time step.
\begin{equation}
	\ket{\psi(\tau_c + \Delta \tau)} =  e^{- E_0 \Delta \tau} \ket{\phi_n} 
\end{equation}
We observe that the ground state energy $E_0$ can be found by
\begin{equation}
	E_0 = - \frac{\log \braketnorm{\psi(\tau_c + \Delta \tau)}}{2 \Delta \tau}
\end{equation}


\section{Finding the ground state by expectation values}
An alternative to using the diffusion propagator to find the ground state energy directly, one
could find the energy by calculating the expectation value of the Hamiltonian. Luckily, it is
relatively easy to modify an existing split step TDSE propagator to calculate the expectation 
value instead of propagating one timestep. 

Recall that a split step propagator divides the problem into parts $H_n$ which we can find an expression
for the matrix exponential $e^{- \imath H \Delta t}$ that is easy to evaluate. This is usually done by 
finding a basis $X_n$ with eigenvalues $\Lambda_n$ that diagonalizes $H_n$. We can then write the 
matrix exponential as
\begin{equation}
	e^{- \imath H \Delta t} = X_n e^{- \imath \Lambda_n \Delta t} X_n^{-1}
\end{equation}
Propagating one timestep with the split step propagator (using 1. order splitting) can then be written
\begin{equation}
	\ket{\psi(t + \Delta t)} = \left ( \prod_{n} X_n e^{- \imath \Lambda_n \Delta t} X_n^{-1} \right ) \ket{\psi(t)} 
\end{equation}
Where one applies the sub operators consecutively. 

To calculate the expectation value of the energy, one should calculate the following quantity
\begin{equation}
	E = \bra{\psi} H \ket{\psi}
\end{equation}
If we split $H$ in the same sub operators as for the propagator, we can write the energy
\begin{equation}
	E = \sum_n \bra{\psi} H_n \ket{\psi} = \sum_n \bra{\psi} X_n \Lambda_n X_n^{-1} \ket{\psi}
\end{equation}
Thus in order to use the split step propagator to calculate the energy expectation value, we have to make
three modifications

\begin{itemize}
\item Modiyfy each sub propagator to apply the diagonalized operators directly, instead
of the matrix exponential
\item Calculate the projection of the propagated state onto the original state at each sub operator.
\item Sum the energies calculated by each sub operator to the total energy
\end{itemize}

\end{document}
